\documentclass[11pt]{article}

%%% Packages
\usepackage{amsfonts}
\usepackage{amsmath}
\usepackage[dvipsnames]{xcolor} % used for notes and solutions
\usepackage{hyperref} % used for links
\hypersetup{
    colorlinks=true,
    linkcolor=blue,
    filecolor=magenta,      
    urlcolor=cyan,
    pdftitle={Overleaf Example},
    pdfpagemode=FullScreen,
    }

\pagestyle{myheadings}
\markright{MIT 18.404/6.840 \hfill PSET 2}
\pagenumbering{gobble}

\usepackage{geometry}
\geometry{
    left=1in,
    right=1in,
    top=1in,
    bottom=1in
}

%%% Formatting

\setlength{\parskip}{\medskipamount}
\setlength{\parindent}{0in}

%%% Useful Commands

\newcommand\bit{\{0, 1\}}

\newcommand\false{\textbf{FALSE}}
\newcommand\true{\textbf{TRUE}}

\newcommand\size[1]{\left|#1\right|} % cardinality
\newcommand\union{\cup}
\newcommand\intersect{\cap}

\newcommand{\F}{\mathbb{F}}
\newcommand{\np}{\mathop{\rm NP}}
\newcommand{\Z}{{\mathbb Z}}
\newcommand{\vol}{\mathop{\rm Vol}}
\newcommand{\conp}{\mathop{\rm co-NP}}
\newcommand{\atisp}{\mathop{\rm ATISP}}
\renewcommand{\vec}[1]{{\mathbf #1}}
\newcommand{\cupdot}{\mathbin{\mathaccent\cdot\cup}}
\newcommand{\mmod}[1]{\ (\mathrm{mod}\ #1)} 

%%% Notes

\newenvironment{hint}{\itshape\color{gray}\textbf{Hint:}}{}
\newcommand\todo[1]{\textbf{\color{red}[[TODO: \textit{#1}]]}}
\newcommand\idk{\textbf{\color{orange}I don't know }}
\newcommand\bonus[1]{BONUS #1}

%%% Questions

%% TODO: Fix \hfill error
\newcommand\thequestion{\thesection}
\newenvironment{question}[2]
{\newpage\section{#1\texorpdfstring{\hfill}{horizontal spacing}{\rm\normalsize #2}}}{}

\newcommand\thesubquestion{\thesubsection}
\newenvironment{subquestion}[2]
{\subsection{#1\texorpdfstring{\hfill}{horizontal spacing}{\rm\normalsize #2}}}{}

\newenvironment{solution}
{\textbf{Solution: }\color{blue}}
{\color{black}}

%%% Assignment

\begin{document}

%%%%%%%%%%%%%%%%%%%%%%%%%%%%%%%%%%
%           Question 1
%%%%%%%%%%%%%%%%%%%%%%%%%%%%%%%%%%

\begin{question}{Book 2.32}{[proving non-CFL]}

% Let \(\Sigma = \{1,2,3,4\}\) and \(C=\{w \in \Sigma^* | \text{ in } w, \text{ the number of 1s equals the number of 2s, and the number of 3s equals the number of 4s }\}\). Show that \(C\) is not context-free.

Let \(\Sigma = \{1,2,3,4\}\) and \(C=\{w \in \Sigma^* | \text{ in } w, \text{ the number of 1s equals the number of 2s, and the }\) \(\text{number of 3s equals the number of 4s }\}\). Show that \(C\) is not context-free.

(Note that the result of this problem demonstrates that the class of CFLs isn’t closed under either complement or intersection. Check that you understand why.)

\begin{solution}



\end{solution}
\end{question}

%%%%%%%%%%%%%%%%%%%%%%%%%%%%%%%%%%
%           Question 2
%%%%%%%%%%%%%%%%%%%%%%%%%%%%%%%%%%

\begin{question}{Book 3.12}{[left reset TM]}



\begin{solution}



\end{solution}
\end{question}

%%%%%%%%%%%%%%%%%%%%%%%%%%%%%%%%%%
%           Question 3
%%%%%%%%%%%%%%%%%%%%%%%%%%%%%%%%%%

\begin{question}{Book 3.18}{[decidable iff enumerable in lex order]}



\begin{solution}



\end{solution}
\end{question}

%%%%%%%%%%%%%%%%%%%%%%%%%%%%%%%%%%
%           Question 4
%%%%%%%%%%%%%%%%%%%%%%%%%%%%%%%%%%

\begin{question}{Book 4.17}{[projection of decidable iff T-recognizable]}



\begin{solution}



\end{solution}
\end{question}

%%%%%%%%%%%%%%%%%%%%%%%%%%%%%%%%%%
%           Question 5
%%%%%%%%%%%%%%%%%%%%%%%%%%%%%%%%%%

\begin{question}{Book 4.22}{[useless states in a PDA]}



\begin{solution}



\end{solution}
\end{question}

%%%%%%%%%%%%%%%%%%%%%%%%%%%%%%%%%%
%           Question 6
%%%%%%%%%%%%%%%%%%%%%%%%%%%%%%%%%%

\begin{question}{Book 4.25}{[decidable DFA problem]}



\begin{solution}



\end{solution}
\end{question}

%%%%%%%%%%%%%%%%%%%%%%%%%%%%%%%%%%
%           Question 7
%%%%%%%%%%%%%%%%%%%%%%%%%%%%%%%%%%

\begin{question}{Book 2.22}{[tricky CFL]}



\begin{solution}



\end{solution}
\end{question}

\end{document}