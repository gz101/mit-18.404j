\documentclass[11pt]{article}

%%% Packages
\usepackage{amsfonts}
\usepackage{amsmath}
\usepackage[dvipsnames]{xcolor} % used for notes and solutions
\usepackage{hyperref} % used for links
\hypersetup{
    colorlinks=true,
    linkcolor=blue,
    filecolor=magenta,      
    urlcolor=cyan,
    pdftitle={Overleaf Example},
    pdfpagemode=FullScreen,
    }

\pagestyle{myheadings}
\markright{MIT 18.404/6.840 \hfill PSET 1}
\pagenumbering{gobble}

\usepackage{geometry}
\geometry{
    left=1in,
    right=1in,
    top=1in,
    bottom=1in
}

%%% Formatting

\setlength{\parskip}{\medskipamount}
\setlength{\parindent}{0in}

%%% Useful Commands

\newcommand\bit{\{0, 1\}}

\newcommand\false{\textbf{FALSE}}
\newcommand\true{\textbf{TRUE}}

\newcommand\size[1]{\left|#1\right|} % cardinality
\newcommand\union{\cup}
\newcommand\intersect{\cap}

\newcommand{\F}{\mathbb{F}}
\newcommand{\np}{\mathop{\rm NP}}
\newcommand{\Z}{{\mathbb Z}}
\newcommand{\vol}{\mathop{\rm Vol}}
\newcommand{\conp}{\mathop{\rm co-NP}}
\newcommand{\atisp}{\mathop{\rm ATISP}}
\renewcommand{\vec}[1]{{\mathbf #1}}
\newcommand{\cupdot}{\mathbin{\mathaccent\cdot\cup}}
\newcommand{\mmod}[1]{\ (\mathrm{mod}\ #1)} 

%%% Notes

\newenvironment{hint}{\itshape\color{gray}\textbf{Hint:}}{}
\newcommand\todo[1]{\textbf{\color{red}[[TODO: \textit{#1}]]}}
\newcommand\idk{\textbf{\color{orange}I don't know }}
\newcommand\bonus[1]{BONUS #1}

%%% Questions

%% TODO: Fix \hfill error
\newcommand\thequestion{\thesection}
\newenvironment{question}[2]
{\newpage\section{#1\texorpdfstring{\hfill}{horizontal spacing}{\rm\normalsize #2}}}{}

\newcommand\thesubquestion{\thesubsection}
\newenvironment{subquestion}[2]
{\subsection{#1\texorpdfstring{\hfill}{horizontal spacing}{\rm\normalsize #2}}}{}

\newenvironment{solution}
{\textbf{Solution: }\color{blue}}
{\color{black}}

%%% Assignment

\begin{document}

%%%%%%%%%%%%%%%%%%%%%%%%%%%%%%%%%%
%           Question 1
%%%%%%%%%%%%%%%%%%%%%%%%%%%%%%%%%%

\begin{question}{Book 1.32}{[parallel addition]}
Let \(\Sigma_3\) be defined as the set of column vectors with three entries, where all entry values are either 0 or 1. A string of symbols in \(\Sigma_3\) gives three rows of 0s and 1s. Consider each row to be a binary number and let
\[B = \{w \in \Sigma_{3}^{*} \text{ } | \text{ the bottom row of } w \text{ is the sum of the top two rows in binary}\}.\]

Show that \(B\) is regular. (Hint: Working with \(B^{R}\) is easier, where \(w^{R}\) is the reverse of the string \(w\). You may assume that if \(w\) is regular, then \(w^{R}\) is regular).

\begin{solution}

To prove that \(B\) is regular, we will instead demonstrate that \(B^{R}\) is regular, utilizing the provided hint. Since \(B^{R}\) consists of strings where the rightmost column is read first (considering \(B\)), we can define a DFA to process \(B^{R}\). By proving that \(B^{R}\) is regular, and given that the reverse of a regular language is regular, we can conclude that \(B\) is also regular.

We first define our DFA, \(D\) as follows. Our alphabet is defined as \(\Sigma = \{000, 011, 101, 110\}\), which represents the only four valid column configurations where the last digit is the sum of the first two. The states of the \(D\) can be written as \(Q = \{q_0, q_1\}\), where \(q_0\) is the state when there is no carry from the previous column, and \(q_1\) is the state when there is a carry present from the previous column.

We define our transition function \(\delta\) to be
\begin{itemize}
    \item From \(q_0\):
    \begin{itemize}
        \item on 000: stay in \(q_0\), since this represents \(0+0=0\) with no carries.
        \item on 011: stay in \(q_0\), since this represents \(0+1=1\) with no carries.
        \item on 101: stay in \(q_0\), since this represents \(1+0=1\) with no carries.
        \item on 110: move to \(q_1\), since this represents \(1+1=0\) with a carry of 1.
    \end{itemize}
    \item From \(q_1\):
    \begin{itemize}
        \item on 011: move to \(q_0\), since this represents \(0+1=1\), with an additional carry.
        \item on 101: move to \(q_0\), since this represents \(1+0=1\), with an additional carry.
        \item on 110: stay in \(q_1\), since this represents \(1+1=0\), with an additional carry.
    \end{itemize}
\end{itemize}

Naturally, our start state is \(q_0\), and our set of accept states is \(F=\{q_0\}\). This is because we start with no carries, and a valid configuration of \(\Sigma_{3}^{*}\) must also have no carries.

Since \(D\) can recognize strings in \(B^{R}\) by processing each column vector from right to left (considering the string in \(B\)). Thus, \(B^{R}\) is regular. Given that the reverse of a regular language is regular, \(B\) is also regular. This completes our proof.

\end{solution}
\end{question}

%%%%%%%%%%%%%%%%%%%%%%%%%%%%%%%%%%
%           Question 2
%%%%%%%%%%%%%%%%%%%%%%%%%%%%%%%%%%

\begin{question}{Book 1.41}{[regular closure under perfect shuffle]}


\begin{solution}

\end{solution}
\end{question}

%%%%%%%%%%%%%%%%%%%%%%%%%%%%%%%%%%
%           Question 3
%%%%%%%%%%%%%%%%%%%%%%%%%%%%%%%%%%

\begin{question}{Book 1.53}{[sequential addition]}


\begin{solution}

\end{solution}
\end{question}

%%%%%%%%%%%%%%%%%%%%%%%%%%%%%%%%%%
%           Question 4
%%%%%%%%%%%%%%%%%%%%%%%%%%%%%%%%%%

\begin{question}{Book 1.60}{[small NFA for: \{\(w | w\) contains an a which exactly \(k\) places from the end\}]}


\begin{solution}

\end{solution}
\end{question}

%%%%%%%%%%%%%%%%%%%%%%%%%%%%%%%%%%
%           Question 5
%%%%%%%%%%%%%%%%%%%%%%%%%%%%%%%%%%

\begin{question}{Book 1.61}{[any DFA for above language must be large]}


\begin{solution}

\end{solution}
\end{question}

%%%%%%%%%%%%%%%%%%%%%%%%%%%%%%%%%%
%           Question 6
%%%%%%%%%%%%%%%%%%%%%%%%%%%%%%%%%%

\begin{question}{Book 2.24}{[very tricky CFL]}
Think about this language in a different way.


\begin{solution}

\end{solution}
\end{question}

%%%%%%%%%%%%%%%%%%%%%%%%%%%%%%%%%%
%           Question 7
%%%%%%%%%%%%%%%%%%%%%%%%%%%%%%%%%%

\begin{question}{Book 2.27}{[if-then-else ambiguous grammar]}
You do not need to prove your grammar works, but adding a few comments about why it does work might help the grader.


\begin{solution}

\end{solution}
\end{question}

%%%%%%%%%%%%%%%%%%%%%%%%%%%%%%%%%%
%           Question 8
%%%%%%%%%%%%%%%%%%%%%%%%%%%%%%%%%%

\begin{question}{(optional) Book 1.59}{[synchronizing sequence]}


\begin{solution}

\end{solution}
\end{question}

\end{document}