\documentclass[11pt]{article}

%%% Packages
\usepackage{amsfonts}
\usepackage{amsmath}
\usepackage[dvipsnames]{xcolor} % used for notes and solutions
\usepackage{hyperref} % used for links
\hypersetup{
    colorlinks=true,
    linkcolor=blue,
    filecolor=magenta,      
    urlcolor=cyan,
    pdftitle={Overleaf Example},
    pdfpagemode=FullScreen,
    }

\pagestyle{myheadings}
\markright{MIT 18.404/6.840 \hfill PSET 1}
\pagenumbering{gobble}

\usepackage{geometry}
\geometry{
    left=1in,
    right=1in,
    top=1in,
    bottom=1in
}

%%% Formatting

\setlength{\parskip}{\medskipamount}
\setlength{\parindent}{0in}

%%% Useful Commands

\newcommand\bit{\{0, 1\}}

\newcommand\false{\textbf{FALSE}}
\newcommand\true{\textbf{TRUE}}

\newcommand\size[1]{\left|#1\right|} % cardinality
\newcommand\union{\cup}
\newcommand\intersect{\cap}

\newcommand{\F}{\mathbb{F}}
\newcommand{\np}{\mathop{\rm NP}}
\newcommand{\Z}{{\mathbb Z}}
\newcommand{\vol}{\mathop{\rm Vol}}
\newcommand{\conp}{\mathop{\rm co-NP}}
\newcommand{\atisp}{\mathop{\rm ATISP}}
\renewcommand{\vec}[1]{{\mathbf #1}}
\newcommand{\cupdot}{\mathbin{\mathaccent\cdot\cup}}
\newcommand{\mmod}[1]{\ (\mathrm{mod}\ #1)} 

%%% Notes

\newenvironment{hint}{\itshape\color{gray}\textbf{Hint:}}{}
\newcommand\todo[1]{\textbf{\color{red}[[TODO: \textit{#1}]]}}
\newcommand\idk{\textbf{\color{orange}I don't know }}
\newcommand\bonus[1]{BONUS #1}

%%% Questions

%% TODO: Fix \hfill error
\newcommand\thequestion{\thesection}
\newenvironment{question}[2]
{\newpage\section{#1\texorpdfstring{\hfill}{horizontal spacing}{\rm\normalsize #2}}}{}

\newcommand\thesubquestion{\thesubsection}
\newenvironment{subquestion}[2]
{\subsection{#1\texorpdfstring{\hfill}{horizontal spacing}{\rm\normalsize #2}}}{}

\newenvironment{solution}
{\textbf{Solution: }\color{blue}}
{\color{black}}

%%% Assignment

\begin{document}

%%%%%%%%%%%%%%%%%%%%%%%%%%%%%%%%%%
%           Question 1
%%%%%%%%%%%%%%%%%%%%%%%%%%%%%%%%%%

\begin{question}{Book 1.32}{[parallel addition]}
Let \(\Sigma_3\) be defined as the set of column vectors with three entries, where all entry values are either 0 or 1. A string of symbols in \(\Sigma_3\) gives three rows of 0s and 1s. Consider each row to be a binary number and let
\[B = \{w \in \Sigma_{3}^{*} \text{ } | \text{ the bottom row of } w \text{ is the sum of the top two rows in binary}\}.\]

Show that \(B\) is regular. (Hint: Working with \(B^{R}\) is easier, where \(w^{R}\) is the reverse of the string \(w\). You may assume that if \(w\) is regular, then \(w^{R}\) is regular).

\begin{solution}

To prove that \(B\) is regular, we will instead demonstrate that \(B^{R}\) is regular, utilizing the provided hint. Since \(B^{R}\) consists of strings where the rightmost column is read first (considering \(B\)), we can define a DFA to process \(B^{R}\). By proving that \(B^{R}\) is regular, and given that the reverse of a regular language is regular, we can conclude that \(B\) is also regular.

We first define our DFA, \(D\) as follows. Our alphabet is defined as \(\Sigma = \{000, 011, 101, 110\}\), which represents the only four valid column configurations where the last digit is the sum of the first two. The states of the \(D\) can be written as \(Q = \{q_0, q_1\}\), where \(q_0\) is the state when there is no carry from the previous column, and \(q_1\) is the state when there is a carry present from the previous column.

We define our transition function \(\delta\) to be
\begin{itemize}
    \item From \(q_0\):
    \begin{itemize}
        \item on 000: stay in \(q_0\), since this represents \(0+0=0\) with no carries.
        \item on 011: stay in \(q_0\), since this represents \(0+1=1\) with no carries.
        \item on 101: stay in \(q_0\), since this represents \(1+0=1\) with no carries.
        \item on 110: move to \(q_1\), since this represents \(1+1=0\) with a carry of 1.
    \end{itemize}
    \item From \(q_1\):
    \begin{itemize}
        \item on 011: move to \(q_0\), since this represents \(0+1=1\), with an additional carry.
        \item on 101: move to \(q_0\), since this represents \(1+0=1\), with an additional carry.
        \item on 110: stay in \(q_1\), since this represents \(1+1=0\), with an additional carry.
    \end{itemize}
\end{itemize}

Naturally, our start state is \(q_0\), and our set of accept states is \(F=\{q_0\}\). This is because we start with no carries, and a valid configuration of \(\Sigma_{3}^{*}\) must also have no carries.

Since \(D\) can recognize strings in \(B^{R}\) by processing each column vector from right to left (considering the string in \(B\)). Thus, \(B^{R}\) is regular. Given that the reverse of a regular language is regular, \(B\) is also regular. This completes our proof.

\end{solution}
\end{question}

%%%%%%%%%%%%%%%%%%%%%%%%%%%%%%%%%%
%           Question 2
%%%%%%%%%%%%%%%%%%%%%%%%%%%%%%%%%%

\begin{question}{Book 1.41}{[regular closure under perfect shuffle]}
For languages \(A\) and \(B\), let the \textbf{perfect shuffle} pf \(A\) and \(B\) be the language
\[\{w | w = a_1 b_1 \ldots a_k b_k, \text{ where } a_1 \ldots a_k \in A \text{ and } b_1 \ldots b_k \in B, \text{ each } a_i,b_i \in \Sigma\}\]

Show that the class of regular languages is closed under perfect shuffle.

\begin{solution}

To show that the class of regular languages is closed under the perfect shuffle operation, we need to demonstrate that if \(A\) and \(B\) are regular languages, then their perfect shuffle \(pf(A,B)\) is also a regular language.

Given that \(A\) and \(B\) are regular, there must exist DFAs \(M_A = (Q_A, \Sigma, \delta_A, q_{0A}, F_A)\) and \(M_B = (Q_B, \Sigma, \delta_B, q_{0B}, F_B)\) that recognize \(A\) and \(B\), respectively.

Let us construct a new DFA \(M_{pf}\) that recognizes \(pf(A,B)\). The idea is to simulate both \(M_A\) and \(M_B\) simultaneously, by walking through the shuffled string in a synchronized manner.

We define our \(M_{pf}\) as follows. Let the states be \(Q_{pf} = Q_A \times Q_B\), such that each state in \(M_{pf}\) corresponds to a pair of states, one from \(M_A\) and one from \(M_B\). We define the alphabet of \(M_{pf}\) to be \(\Sigma_{pf} = \Sigma \times \Sigma\). The transition function is \(\delta_{pf}((p,q),(a,b)) = (\delta_A(p,a), \delta_B(q,b))\), where \((p,q) \in Q_{pf}\) and \(a,b \in \Sigma\). Finally, our start state is \(q_{0pf} = (q_{0A},q_{0B})\), and the set of accepting states are \(F_{pf} = \{(p,q) | p \in F_A \text{ and } q \in F_B\}\).

For a string \(w = a_1 b_1 \ldots a_k b_k\) in the perfect shuffle of \(A\) and \(B\), \(M_{pf}\) starts in state \(q_{0pf}\) and processes the pairs \(a_i,b_i\) using the transition function. It will end up in an accepting state if and only if the \(a_i\) sequence leads \(M_A\) to an accepting state and the \(b_i\) sequence leads \(M_B\) to an accepting state.

Thus, \(M_{pf}\) recognizes \(pf(A,B)\) and since \(pf(A,B)\) can be recognized by a DFA, it is regular. This means that the class of regular languages is closed under the perfect shuffle operation, and this concludes our proof.

\end{solution}
\end{question}

%%%%%%%%%%%%%%%%%%%%%%%%%%%%%%%%%%
%           Question 3
%%%%%%%%%%%%%%%%%%%%%%%%%%%%%%%%%%

\begin{question}{Book 1.53}{[sequential addition]}
Let \(\Sigma = \{0,1,+,=\}\) and
\[ADD = \{x=y+z | x,y,z \text{ are binary integers, and } x \text{ is the sum of } y \text{ and } z\}.\]

Show that \(ADD\) is not regular.

\begin{solution}

To show that \(ADD\) is not regular, we use the Pumping Lemma for regular languages. We assume for the sake of contradiction that \(ADD\) is regular. Then, there exists a pumping length \(p\). Let us choose the string \(s\) to be \(10^p = 1^p + 1^p\), which is a string that represents a valid addition in binary.

Now let us apply the Pumping Lemma. Given the structure of \(s\), if we write \(s=xyz\), where the conditions from the Pumping Lemma hold, the pump-able substring \(y\) can have several forms:
\begin{enumerate}
    \item \(y\) contains only 1s from the leftmost section (before the \(=\) sign): \(y=1^k\) for some \(k>0\).
    \item \(y\) contains only 0s: \(y=0^k\) for some \(k \leq p\).
    \item \(y\) spans across the \(=\) or \(+\) signs. But this cannot happen due to the Pumping Lemma restriction that \(|xy| \leq p\).
    \item \(y\) contains only 1s from the section after the \(+\) sign: \(y=1^k\) for some \(k>0\).
\end{enumerate}

Now consider the pumped string \(xy^2z\) for each scenario. We omit the third scenario from consideration as it cannot occur.

If \(y\) contains only 1s from the leftmost section, then the pumped string will increase the value on the left side of the equation without changing the right side, resulting in an invalid equation. For the second scenario, if \(y\) consists of only 0s, then the pumped string will again increase the value on the left side of the equation without changing the right side, leading to an invalid equation. Finally, if \(y\) contains only 1s from the section after the \(+\) sign, then the pumped string will increase one of the values on the right side of the equation without changing the left side, leading to an invalid equation.

In all the possible configurations for \(y\), pumping \(y\) to form \(xy^2z\) results in a string that is not in \(ADD\). This is a contradiction, and thus our initial assumption that \(ADD\) is regular must be false. Therefore, this concludes our proof that \(ADD\) is not regular.

\end{solution}
\end{question}

%%%%%%%%%%%%%%%%%%%%%%%%%%%%%%%%%%
%           Question 4
%%%%%%%%%%%%%%%%%%%%%%%%%%%%%%%%%%

\begin{question}{Book 1.60}{[small NFA for: \{\(w | w\) contains an \textbf{a} which exactly \(k\) places from the end\}]}

Let \(\Sigma = \{a,b\}\). For each \(k \geq 1\), let \(C_k\) be the language consisting of all strings that contain an \(a\) exactly \(k\) places from the right-hand end. Thus, \(C_k = \Sigma^* a \Sigma^{k-1}\). Describe an NFA with \(k+1\) states that recognizes \(C_k\) in terms of both a state diagram and a formal description.

\begin{solution}

Given \(k \geq 1\), we want to design an NFA to accept strings where there is an \(a\) exactly \(k\) positions from the end of the string. This means that the NFA will need to use non-deterministic branching every time it sees an \(a\) and continue transitioning for \(k-1\) more characters, at which point it should be in the accept state if the input string is in the language. 

Non-determinism is required since it is not known if the encountered \(a\) is the one that is \(k-1\) steps from the end of the string. If the NFA reaches the end of the input string before this, then that branch of the computation rejects. If at least one branch accepts, the NFA accepts, otherwise it rejects the input string.

We formally describe the NFA below.
\begin{itemize}
    \item \textbf{States}: \(Q=\{q_0,q_1,\ldots,q_{k+1}\}\)
    \item \textbf{Alphabet}: \(\Sigma = \{a,b\}\)
    \item \textbf{Transitions}:
    \begin{itemize}
        \item \(\delta(q_0,a) = \{q_0, q_1\}\) (the read \(a\) could be one that is \(k-1\) steps from the end, or not).
        \item \(\delta(q_0,b) = \{q_0\}\) (we ignore characters until we see the first occurrence of \(a\)).
        \item \(\delta(q_1,a) = \{q_1,q_2\}\) (either this \(a\) or the previous one could be \(k-1\) steps from the end).
        \item \(\delta(q_1,b) = \{q_2\}\) (if \(k>1\)) (it would be an accepting state if \(k=1\)).
        \item For \(2 \leq i < k\):
        \begin{itemize}
            \item \(\delta(q_i,a)=\{q_1,q_{i+1}\}\) (the read \(a\) could be one that is \(k-1\) steps from the end, or it is part of the final \(k-1\) characters).
            \item \(\delta(q_i,b)=\{q_{i+1}\}\) (the input forms part of the final \(k-1\) characters).
        \end{itemize}
        \item \(\delta(q_k,a)= \delta(q_k,b) = \{q_{k+1}\}\) (always accepts if the final character is valid).
    \end{itemize}
    \item \textbf{Start State}: \(q_0\)
    \item \textbf{Accept State}: \(q_{k+1}\)
\end{itemize}

A state diagram follows naturally from the above description.

\end{solution}
\end{question}

%%%%%%%%%%%%%%%%%%%%%%%%%%%%%%%%%%
%           Question 5
%%%%%%%%%%%%%%%%%%%%%%%%%%%%%%%%%%

\begin{question}{Book 1.61}{[any DFA for above language must be large]}

Consider the languages \(C_k\) defined in Problem 1.60. Prove that for each \(k\), no DFA can recognize \(C_k\) with fewer than \(2^k\) states.

\begin{solution}

Suppose for the sake of contradiction that there exists a DFA \(D\) that recognizes \(C_k\) and has fewer than \(2^k\) states. Consider running the DFA \(D\) on all strings of length \(k\). Since there are \(2^k\) such strings and \(D\) has fewer than \(2^k\) states, by the pigeonhole principle, there must be at least two different strings, say \(w_1\) and \(w_2\), that cause \(D\) to end up in the same state after processing the strings.

The state must have the same behavior regarding any further inputs that are the same, since the DFA doesn't have memory of past inputs beyond its current state. Consider the following possible cases if the strings were extended in length (i.e. \(w_1 b^{k-1}\) and \(w_2 b^{k-1}\)).

If both \(w_1\) and \(w_2\) end in an \(a\), appending \(b^{k-1}\) to both will result in both \(w_1 b^{k-1}\) and \(w_2 b^{k-1}\) being in \(C_k\). \(D\) would accept both strings.

If both \(w_1\) and \(w_2\) do not end in an \(a\), appending \(b^{k-1}\) to both will result in neither of the strings being in \(C_k\). \(D\) would reject both strings.

Now suppose without loss of generality that \(w_1\) ends in an \(a\) and \(w_2\) doesn't. If we append \(b^{k-1}\) to both, \(w_1 b^{k-1}\) is in \(C_k\), but \(w_2 b^{k-1}\) isn't. However, since \(D\) is in the same state after reading both \(w_1\) and \(w_2\), it cannot differentiate between these two cases and will either accept both strings or reject both, which gives our desired contradiction.

Thus, no DFA can recognize \(C_k\) with fewer than \(2^k\) states.

\end{solution}
\end{question}

%%%%%%%%%%%%%%%%%%%%%%%%%%%%%%%%%%
%           Question 6
%%%%%%%%%%%%%%%%%%%%%%%%%%%%%%%%%%

\begin{question}{Book 2.24}{[very tricky CFL]}

Let \(E = \{ a^i b^j | i \neq j \text{ and } 2i \neq j \}\). Show that \(E\) is a context-free language.

Hint: Think about this language in a different way.

\begin{solution}

To show that a language is context-free, one common technique is to find a CFG that generates it. Given \(E\) as defined in the problem statement, we use the hint and consider what the language is not. Since the language of \(E\) is defined by what it is not equal to, we consider the complement \(\overline{E}\), which contains the strings where \(i = j\) or \(2i = j\).

\(\overline{E}\) can be described by the languages \(F_1 = \{a^i b^i | i \geq 0\}\) (where \(i=j\)), and \(F_2 = \{a^i b^{2i} | i \geq 0\}\) (where \(2i=j\)). It is trivial to show that the CFG defined by \(S_1 \rightarrow a S_1 b | \epsilon\) generates \(F_1\), and the CFG defined by \(S_2 \rightarrow a S_2 bb | \epsilon\) generates \(F_2\).

Consider the regular language \(R=a^* b^*\). This is the set of all strings with any number of \(a\)s followed by any number of \(b\)s. \(E\) consists of strings in \(R\) that are not in \(F_1 \union F_2\). This can be framed as a set difference of \(R \setminus (F_1 \union F_2)\). While CFLs aren't closed under set difference, we are performing a set difference with a regular language, which has more flexible properties.

Given that regular languages are closed under complementation and CFLs are closed under intersection with regular languages, the set difference of a regular language and CFL can be equivalently represented by an intersection of the complement of the CFL (within the regular language) and the regular language itself.

So consider the complement of \(F_1 \union F_2\) within \(R\). This complement is regular since it is defined within the regular language \(R\). Then \(E\) can be equivalently defined as \(R \intersect (\overline{R} \intersect (\overline{F_1 \union F_2}))\). To elaborate, \(\overline{R}\) is the complement of \(R\) within the universe of all strings. Since \(\overline{R}\) is also a regular language (because regular languages are closed under complementation) and the intersection of a regular language and CFL is a CFL, the resulting language is context-free.

Since \(R\) is a regular language and \(\overline{R} \intersect (\overline{F_1 \union F_2})\) is context-free, so then \(E\) is a context-free language. This concludes our proof.

\end{solution}
\end{question}

%%%%%%%%%%%%%%%%%%%%%%%%%%%%%%%%%%
%           Question 7
%%%%%%%%%%%%%%%%%%%%%%%%%%%%%%%%%%

\begin{question}{Book 2.27}{[if-then-else ambiguous grammar]}

Let \(G = (V, \Sigma, R, \langle STMT \rangle)\) be the following grammar.

\begin{equation*}
    \begin{split}
        \langle STMT \rangle &\rightarrow \langle ASSIGN \rangle | \langle IF-THEN \rangle | \langle IF-THEN-ELSE \rangle \\
        \langle IF-THEN \rangle &\rightarrow \text{if condition then } \langle STMT \rangle \\
        \langle IF-THEN-ELSE \rangle &\rightarrow \text{if condition then } \langle STMT \rangle \text{ else } \langle STMT \rangle \\
        \langle ASSIGN \rangle &\rightarrow \text{ a := 1 }
    \end{split}
\end{equation*}

\begin{equation*}
    \begin{split}
        \Sigma &= \{ \text{if, condition, then, else, a:= 1} \} \\
        V &= \{ \langle STMT \rangle, \langle IF-THEN \rangle, \langle IF-THEN-ELSE \rangle, \langle ASSIGN \rangle \}
    \end{split}
\end{equation*}

\(G\) is a natural-looking grammar for a fragment of a programming language, but \(G\) is ambiguous.
\begin{enumerate}
    \item Show that \(G\) is ambiguous.
    \item Give a new unambiguous grammar for the same language.
\end{enumerate}

You do not need to prove your grammar works, but adding a few comments about why it does work might help the grader.

\begin{solution}

To show that \(G\) is ambiguous, we need to prove that there exists a string that can be derived in two different parse trees. The ambiguity int his grammar arises due to the ``dangling else" problem. Let us consider the following statement: \verb|if condition then if condition then a:=1 else a:= 1|. This statement can be interpreted in the two following ways:

\begin{enumerate}
    \item The ``else" belongs to the first ``if":
    \begin{equation*}
        \begin{split}
            \langle STMT \rangle &\rightarrow \langle IF-THEN-ELSE \rangle \\
                                 &\rightarrow \text{ if condition then } \langle STMT \rangle \text{ else } \langle STMT \rangle \\
                                 &\rightarrow \text{ if condition then } \langle IF-THEN \rangle \text{ else } \langle STMT \rangle \\
                                 &\rightarrow \text{ if condition then (if condition then } \langle STMT \rangle \text{) else } \langle STMT \rangle \\
                                 &\rightarrow \text{ if condition then (if condition then a := 1 ) else a := 1}
        \end{split}
    \end{equation*}
    \item The ``else" belongs to the second ``if":
    \begin{equation*}
        \begin{split}
            \langle STMT \rangle &\rightarrow \langle IF-THEN \rangle \\
                                 &\rightarrow \text{ if condition then } \langle STMT \rangle \\
                                 &\rightarrow \text{ if condition then } \langle IF-THEN-ELSE \rangle \\
                                 &\rightarrow \text{ if condition then (if condition then } \langle STMT \rangle \text{ else } \langle STMT \rangle \text{)} \\
                                 &\rightarrow \text{ if condition then (if condition then a := 1 else a := 1)}
        \end{split}
    \end{equation*}
\end{enumerate}

Both grammars are valid according to \(G\), proving that \(G\) is ambiguous.

We now turn to giving a new unambiguous grammar for the same language. We can fix the ambiguity by ensuring that every ``else" clause is associated with the closest preceding ``if" that doesn't already have an ``else". One way to do this is by distinguishing between statements that end with an ``else" clause and those that do not:

\begin{equation*}
    \begin{split}
        \langle STMT \rangle &\rightarrow \langle MATCHED \rangle | \langle UNMATCHED \rangle \\
        \langle MATCHED \rangle &\rightarrow \langle ASSIGN \rangle | \langle MATCHED-IF-THEN-ELSE \rangle \\
        \langle MATCHED-IF-THEN-ELSE \rangle &\rightarrow \text{if condition then } \langle MATCHED \rangle \text{ else } \langle MATCHED \rangle \\
        \langle UNMATCHED \rangle &\rightarrow \text{if condition then } \langle STMT \rangle | \\
        &\text{if condition then } \langle MATCHED \rangle \text{ else } \langle UNMATCHED \rangle \\
        \langle ASSIGN \rangle &\rightarrow \text{ a := 1 }
    \end{split}
\end{equation*}

A \(\langle MATCHED \rangle\) statement is either an assignment or an if-then-else where both the ``then" and ``else" parts are also \(\langle MATCHED \rangle\) statements. On the other hand, an \(\langle UNMATCHED \rangle\) statement is an if-then statement where the ``then" part can be any statement or an if-then-else where the ``then" part is a \(\langle MATCHED \rangle\) statement and the ``else" part is an \(\langle UNMATCHED \rangle\) statement.

This ensures that the ``else" is always matched with the nearest ``if", thereby removing the ambiguity.

\end{solution}
\end{question}

%%%%%%%%%%%%%%%%%%%%%%%%%%%%%%%%%%
%           Question 8
%%%%%%%%%%%%%%%%%%%%%%%%%%%%%%%%%%

\begin{question}{(optional) Book 1.59}{[synchronizing sequence]}


\begin{solution}

\end{solution}
\end{question}

\end{document}